%----------WORK EXPERIENCE
\section{Work Experience}
  \resumeSubHeadingListStart
    %== AMD ===============
    \resumeSubheading
      {Senior Software Development Engineer}{Jan. 2022 -- Present}
      {Advanced Micro Devices, Inc. -- GPU Technologies \& Engineering}{Boxborough, MA}
      \resumeItemListStart
        \resumeItem{
          \textbf{Key contributor to the creation of Radeon GPU Detective (RGD)} by spearheading the design and development of DirectX 12 and Vulkan driver components and collaborating with cross-functional teams on the overall project. Adopted by major video game studios to resolve GPU crashes in over six AAA titles, earning internal recognition with an ``Executive Spotlight'' award.
        }
        \resumeItem{
          \textbf{Architected and led the development of a new GPU profiling and tracing system}, replacing a rigid legacy setup with a centralized, modular, cross-driver solution (DX12, Vulkan, OpenGL, OpenCL, HIP, ROCm) for Windows and Linux. Guided and collaborated with a small team, alongside cross-functional partners, to build a system that is highly extensible and robust. Developed an extensive Google Test suite for rigorous validation of system integrity.
        }
        % \resumeItem{
        %   Co-designed a Rust-based internal package manager, used widely within the G\&E group, for streamlining the distribution of internal applications, libraries, and scripts
        % }
        \resumeItem{
          \textbf{Improved driver developer productivity} by developing internal tooling for deploying and managing AMD graphics \& compute drivers
        }
      \resumeItemListEnd
    %== LANL ==============
    \resumeSubheading
      {Research Technologist}{Mar. 2017 -- Dec. 2021}
      {Los Alamos National Laboratory -- Computational Earth Science group (EES-16)}{Los Alamos, NM}
      \resumeItemListStart
        \resumeItem{
          Project lead and primary developer on \textit{TINerator}, an open-source \textbf{Python} module for generating multi-scale geologic \textbf{3D polygonal meshes} from \textbf{GIS} \& \textbf{geospatial} data for use in flow and transport simulation codes -- widely used across U.S. D.O.E. national laboratories and presented in numerous conferences, lectures, and journals
        }
        \resumeItem{
          Reduced model setup and analysis time by $\sim$70\% by developing an \textbf{award-winning program} for interactive \textbf{2D} \& \textbf{3D visualization} + \textbf{analysis} of geospatial data
        }
        \resumeItem{
          Project lead and primary developer on \textit{VORONOI}, an open-source \textbf{Fortran90} application for \textbf{MPI}-based ``embarrassingly parallel'' generation of Voronoi tessellations from polygonal meshes, formatted for use in various multi-physics numerical models -- presented at several conferences and has national \& international users
        }
        \resumeItem{
          Reduced computation time by $>$100x by leading development on an internal \textbf{Julia} module containing a \textbf{novel algorithm} for parallel computation of fire spread behavior from geospatial data
        }
        \resumeItem{
          Reduced mesh generation time by $\sim$40\% by designing high-performing \textbf{algorithms} in \textbf{C, C++, and Fortran} for polygonal mesh attribute interpolation \& dynamic sub-mesh querying and extraction
        }
        % \resumeItem{
        %   Dramatically increased the portability and stability of a mesh generation codebase by writing \textbf{C/FORTRAN bindings}, adding support for new compilers, improving \textbf{unit} and \textbf{regression} testing, and migrating the build system from Make to \textbf{CMake}
        % }
        %\resumeItem{
        %  Responsible for the end-to-end procurement, setup \& deployment of an \textbf{Ubuntu Linux} web server hosting an internal group wiki
        %}
      \resumeItemListEnd
    %== ASU ===============
    \resumeSubheading
      {Graduate Research Assistant}{Jun. 2015 -- Dec. 2015}
      {Arizona State University -- School of Earth \& Space Exploration (SESE)}{Tempe, AZ}
      \resumeItemListStart
        \resumeItem{
          Developed an \textbf{Artificial Neural Network} which, when deployed on a cluster of drones acting as a ``mesh network'', finds and directs the drones to the optimal spatial configuration for maximum network coverage
        }
        \resumeItem{
          Reduced neural network training \& execution time by $\sim$70\% by implementing \textbf{shared-memory parallelization} with \textbf{OpenMP}
        }
        % \resumeItem{
        %   Further reduced execution time by $\sim$50\%, while also decreasing technical debt and code complexity, by \textbf{profiling} runtime characteristics with Valgrind/Callgrind and \textbf{refactoring} high-cost functions
        % }
        % \resumeItem{
        %   Developed a \textbf{Python} script to visualize neural network behavior by interfacing with \textbf{Blender} to render drone movement in 3D
        % }
      \resumeItemListEnd
  \resumeSubHeadingListEnd
