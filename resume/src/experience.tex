%----------WORK EXPERIENCE
\section{Work Experience}
  \resumeSubHeadingListStart
    \resumeSubheading
      {Research Technologist}{Mar. 2017 -- Present}
      {Los Alamos National Laboratory -- Computational Earth Science group (EES-16)}{Los Alamos, NM}
      \resumeItemListStart

        \resumeItem{
          Project lead and primary developer on \textit{TINerator}, an open-source \textbf{Python} module for generating multi-scale geologic \textbf{3D polygonal meshes} from \textbf{GIS} \& \textbf{geospatial} data for use in flow and transport simulation codes -- widely used across U.S. D.O.E. national laboratories and presented in numerous conferences, lectures, and journals
        }

        \resumeItem{
          Project lead and primary developer on \textit{VORONOI}, an open-source \textbf{Fortran90} application for \textbf{MPI}-based ``embarrassingly parallel'' generation of Voronoi tessellations from polygonal meshes, formatted for use in various multi-physics numerical models -- presented at several conferences and has national \& international users
        }

        \resumeItem{
          Reduced computation time by \textbf{$\gt 100x$} by leading development on an internal \textbf{Julia} module containing a \textbf{novel algorithm} for parallel computation of fire spread behavior from geospatial data
        }

        \resumeItem{
          Reduced mesh generation time by $sim$ 40\% by designing high-performing \textbf{algorithms} in \textbf{C, C++, and Fortran} for polygonal mesh attribute interpolation \& dynamic sub-mesh querying and extraction
        }

        \resumeItem{
          Reduced model setup and analysis time by $\sim 70%$ by developing an \textbf{award-winning program} for interactive \textbf{2D} \& \textbf{3D visualization} + \textbf{analysis} of geospatial data
        }

        \resumeItem{
          Dramatically increased the portability and stability of a mesh generation codebase by writing \textbf{C/FORTRAN bindings}, adding support for new compilers, improving \textbf{unit} and \textbf{regression} testing, and migrating the build system from Make to \textbf{CMake}
        }

        \resumeItem{
          Designed and deployed \textbf{static websites}, \textbf{API documentation}, \textbf{unit tests}, and \textbf{CI/CD} for over five internal codebases
        }

        \resumeItem{
          Responsible for the end-to-end procurement, setup \& deployment of an \textbf{Ubuntu Linux} web server hosting an internal group wiki
        }
        \resumeItemListEnd
      
    \resumeSubheading
      {Patent Analyst}{May 2016 -- Jan. 2017}
      {Global Patent Solutions, LLC}{Scottsdale, AZ}
      \resumeItemListStart
        \resumeItem{
          Performed extensive database research on technical literature related to the patentability of client invention concepts in the domains of nanotechnology, semiconductors, and electromagnetic elements
        }

        \resumeItem{
          Prepared and submitted comprehensive reports on patentability \& technical research findings to clients
        }
      \resumeItemListEnd

    \resumeSubheading
      {Graduate Research Assistant}{Jun. 2015 -- Dec. 2015}
      {Arizona State University -- School of Earth \& Space Exploration (SESE)}{Tempe, AZ}
      \resumeItemListStart
        \resumeItem{
          Developed an \textbf{Artificial Neural Network} which, when deployed on a cluster of drones acting as a ``mesh network'', finds and directs the drones to the optimal spatial configuration for maximum network coverage
        }

        \resumeItem{
          Reduced neural network training \& execution time by $\sim$70\% by implementing \textbf{shared-memory parallelization} with \textbf{OpenMP}
        }

        \resumeItem{
          Further reduced execution time by $\sim$50\%, while also decreasing technical debt and code complexity, by \textbf{profiling} runtime characteristics with Valgrind/Callgrind and \textbf{refactoring} high-cost functions
        }

        \resumeItem{
          Developed a \textbf{Python} script to visualize neural network behavior by interfacing with \textbf{Blender} to render drone movement in 3D
        }
      \resumeItemListEnd
  \resumeSubHeadingListEnd